\chapter{Simspark}

\section{Perceptors}

% Describe in detail, also the message format, see TEXT_INSTEAD... and Oliver's hope-1.pdf manual

% Explain that these perceptors available to all simulations

\subsection{GyroRate Perceptor}

\subsection{HingeJoint Perceptor}

\subsection{Hinge2Joint Perceptor}

\subsection{UniversalJoint Perceptor} 

% Explain that these perceptors are specific to the soccer simulation

\subsection{Vision Perceptor}

% What about the Restricted Vision Perceptor?

% The current soccer bot seems not to use it, still experimental/unsupported?

\subsection{GameState Perceptor}
\subsection{AgentState Perceptor}
\subsection{Hear Perceptor}
\subsection{Touch Perceptor}
\subsection{ForceResistance Perceptor}

\section{Effectors/Actuators}

% Describe in detail, also the message format, see TEXT_INSTEAD...

\section{Create Effector}

When an agent initially connects to the server it is invisible and
cannot take affect a simulation in any meaningful way. It only
possesses a so called CreateEffector.

An agent uses this effector to advice the server to construct it
according to a scene description file it passes as a parameter. This
file is used to construct the physical representation and all further
effectors and preceptors.

After the agent representation is constructed in the server the agent
should do further simulation specific setup. For example in the soccer
simulation each agent is required to register to a team and acquire a
unique player number. For these tasks usually a special effector like
the SoccerInitEffector is used.

% Explain that these effectors are available to all simulations

\subsection{HingeJoint Effector}

\subsection{Hinge2Joint Effector}

\subsection{UniversalJoint Effector}

% Explain that these effectors are soccer specific, and some only in deprecated versions

\subsection{Init Effector}
\subsection{Beam Effector}
\subsection{Drive Effector}

% Only used in rcssserver3D

\subsection{Kick Effector}

% Only used in rcssserver3D

\subsection{Catch Effector}

% Only used in rcssserver3D

\subsection{Say Effector}

\section{Simulation Update Loop}

% I think people want to know how this works, maybe we can give an overview here and go into more detail in the developer manual? What do you think?

SimSpark implements a simple internal event model that immediately
executes every action received from an agent. It does not try to
compensate any network latency or compensate for different computing
resources available to the connected agents.

A consequence is that SimSpark currently does not guarantee that
events are reproducible. This means repeated simulations may have a
different outcome, depending on network delays or load variations on
the machines hosting the agents and the server.

A benefit of the simple structure however are speed gains that make it
interesting for machine learning tasks as in these setups an often
large number of different agent and simulation configurations are
repeatedly tested.

Further the SimSpark main loop is highly customizable as it is
entirely build upon plugins we call simcontrol nodes. Simcontrol nodes
are registered to the simulation server. They act in response to
control events. The simulation server repeatedly generates these as it
executes an abstracted the main loop.

The event types are an 'init' event once when the simulation server
starts and a 'done' event on shutdown. The main then loop cycles
repeatedly through the 'start cycle', 'sense agent', 'act agent' and
'end cycle' events.

Apart from generating control events the simulation server advances
the simulation time passed in the last cycle. Depending on its
configuration it either does this in discrete time quanta or in one
single step.

A simcontrol node further can take responsibility for the time
measurement, for example to synchronize the simulation time with the
real time used to render the scene.  Otherwise the simulation is
stepped a fixed time step as often as possible.

In this way all management tasks are implemented as plugins to the
simulation server. This involves the agent management, monitor
management, rendering, mouse and keyboard input and network code.

This setup allows us to configure the simulation at runtime as either
a monolithic application that does both simulation and rendering or as
a dedicated simulation server that defers rendering to a remote
monitor application.

% Single vs. Multithreaded Runloop (how to setup)

% Yuan agreed to write the next two parts :-)

\subsection{Single-threaded Timer}

\subsection{Multi-threaded Timer}

\section{Setup Scripts}

% describe purpose of scripts in ~/.rcssserver3d/]

% kerosin.rb for rendering configuration of simspark

%%% Local Variables: 
%%% mode: latex
%%% TeX-master: "user-manual"
%%% End: 
