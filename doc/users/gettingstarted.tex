
 \chapter{Getting Started}

\section{Download and Installation Instructions}

%Currently the soccer3d CVS builds only for Linux (the Win32 build system is not yet ported/tested, so we should focus on *nix systems.

This chapter explains how to install SimSpark on an Ubuntu Linux
system\footnote{This guide is adapted from the guide created by the
Little Green Bats Team at
\url{http://www.littlegreenbats.nl/?q=node/70}}. It should work with
slight modifications on other distributions like Fedora, Suse and
especially Debian.

\begin{enumerate}

\item Enable additional repositories

Depending on the distribution version you are using you might need to
enable additional repositories in order to install all required
packages. Please refer to your system specific documentation for
details. In Ubuntu Linux these repositories are called
\texttt{Universe} and
\texttt{Multiverse}

To enable them you have to edit the \texttt{/etc/apt/sources.list}
file and update the local package database.

\texttt{\$ sudo gedit /etc/apt/sources.list}


\texttt{\$ sudo apt-get update}


In RPM based distributions you may need to enable further download
locations or download and install RPM packets manually. There are
specialized search engines for RPM packages available like 

\begin{enumerate}
\item\url{http://packman.links2linux.de/} and
\item\url{http://rpmfind.net/}
\end{enumerate}

\item Install dependencies

SimSpark builds and depends upon the work of other software
projects. We use the Gnu autotools to configure and build
SimSpark. Therefore these packages are required:

\begin{enumerate}
\item autoconf
\item automake
\item libtool
\end{enumerate}

Further required packages are:

\begin{enumerate}
\item Ruby
\item The Open Dynamics Engine (ODE)
\item The boost C++ libraries
\item S-Lang (libslang)
\end{enumerate}

The rendering of SimSpark can be omitted from the build process
therefore these packages are optional:

\begin{enumerate}
\item OpenGL
\item FreeGlut
\item SDL
\item Freetype
\item Developer Image Library (DevIL)
\item Tiff library (libtiff)
\end{enumerate}

In order to build the SimSpark documentation you need a Latex
installation (e.g. \texttt{tetex}).

If you want to build the rsgedit gui you need to install
\texttt{wxWidgets}. The following commands download and installed the required packages:

\texttt{\$ sudo apt-get install g++ ruby1.9 ruby1.9-dev libode0-dev\
libboost-dev libsdl-dev libfreetype6-dev libdevil-dev autoconf\
automake1.9 libtool freeglut3-dev tetex-extra cvs xlibs-dev\
libtiff4-dev libslang1-dev}

\texttt{\$ sudo rm /usr/bin/ruby}


\texttt{\$ sudo ln -s /usr/bin/ruby1.9 /usr/bin/ruby}


\texttt{\$ sudo ln -s /usr/lib/libruby1.9.so /usr/lib/libruby.so}

\item Check out the source from the source forge CVS repository

The SimSpark source is hosted in a CVS repository at
sourceforge.net. In order to build the source first download the
current version, this is called \texttt{check out} in CVS terminology:

\texttt{\$ cvs -d:pserver:anonymous@sserver.cvs.sourceforge.net:/cvsroot/sserver login}


When you get asked for a password, just press enter.


\texttt{\$ cvs -z3 -d:pserver:anonymous@sserver.cvs.sourceforge.net:/cvsroot/sserver co -P rcsoccersim/rcssserver3D}


\item Build and install the soccer server

The automake build tools use a \texttt{configure} script that adapts
the build process to your system allows you to customize it. The
configure script accepts a number of options that you can add to it's
command line:

\begin{enumerate}

\item \texttt{--help} lists all available configure options. There are some more available 
that would exceed the scope of this manual

\item \texttt{--enable-debug=no} builds an optimized version of SimSpark 
that contains no debug symbols

\item \texttt{--enable-kerosin=no} builds SimSpark without rendering support

\item \texttt{--prefix=/some/path} defines the path where the \texttt{make install} will later 
install the SimSpark executable, plugins and resources into your
system. If omitted it defaults to /usr/local

\end{enumerate}

Change in to the top level source directory call the bootstrap script
that invokes the autotools, run configure with your custom options,
start the build process and install the server into your system.

\texttt{\$ cd rcsoccersim/rcssserver3D/}

\texttt{\$ ./bootstrap}

\texttt{\$ ./configure}

\texttt{\$ make}

\texttt{\$ sudo make install}

\item Make sure the linker can find your shared libraries if you changed
the install prefix as described above with the \texttt{--prefix}
option.

\texttt{\$ sudo gedit /etc/ld.so.conf}

Add your install prefix if it isn't already there, save and close.

\texttt{\$ sudo ldconfig}

\item run the simulation

{\texttt{\$ cd}}

{\texttt{\$ simspark}}

\end{enumerate}

\section{An Example of a Simulation Run}

%Take the user through the steps of running the soccer simulation and show what to expect (screen shots)

%I think for the initial outline of the manual it's easier to just focus on either app/simpark app/monitorspark  or rsgedit.

%%% Local Variables: 
%%% mode: latex
%%% TeX-master: "user-manual"
%%% End: 
