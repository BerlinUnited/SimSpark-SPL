 \chapter{Getting Started}

\section{Download and Installation Instructions}

%Currently the soccer3d CVS builds only for Linux (the Win32 build system is not yet ported/tested, so we should focus on *nix systems.

This chapter explains how to install SimSpark on an Ubuntu Linux
system\footnote{This guide is adapted from the guide created by the
  Little Green Bats Team at
  \url{http://www.littlegreenbats.nl/?q=node/70}}. It should work with
slight modifications on other distributions like Fedora, Suse and
especially Debian.

\begin{enumerate}

\item Depending on the distribution version you are using you might need
to enable additional repositories in order to install all required
packages (Please refer to your system specific documentation). In
Ubuntu these repositories are called 'Universe' and 'Multiverse'.

To enable them you have to edit the /etc/apt/sources.list file and
update the local package database.

\$ sudo gedit /etc/apt/sources.list
\$ sudo apt-get update

\item Install dependencies:

SimSpark builds and depends upon the work of other software
projects.

We use the gnu autotools to configure and build SimSpark. Therefore
these packages are required:

autoconf
automake
libtool

Further required, i.e. non optional packages are

Ruby
The Open Dynamics Engine (ODE)
The boost C++ libraries
S-Lang (libslang)

The rendering of SimSpark can be omitted from the build process
therefore these packages are optional

OpenGL
FreeGlut
SDL
Freetype
Developer Image Library (DevIL)
Tiff library libtiff

In order to build the SimSpark documentation you need a Latex
installation (e.g. tetex).

In order to build the rsgedit gui you need to install wxWidgets.

\$ sudo apt-get install g++ ruby1.9 ruby1.9-dev libode0-dev\
libboost-dev libsdl-dev libfreetype6-dev libdevil-dev autoconf\
automake1.9 libtool freeglut3-dev tetex-extra cvs xlibs-dev\
libtiff4-dev libslang1-dev

\$ sudo rm /usr/bin/ruby
\$ sudo ln -s /usr/bin/ruby1.9 /usr/bin/ruby
\$ sudo ln -s /usr/lib/libruby1.9.so /usr/lib/libruby.so

\item Check out the source from the source forge CVS repository:

\$ cvs -d:pserver:anonymous@sserver.cvs.sourceforge.net:/cvsroot/sserver login
[ when asked for a password, just press enter]

\$ cvs -z3 -d:pserver:anonymous@sserver.cvs.sourceforge.net:/cvsroot/sserver co -P rcsoccersim/rcssserver3D

\item Build and install the soccer server

The configure script accepts a number of options that you can add to
it's command line.


--help lists all available configure options

--enable-debug=no builds an optimized version of SimSpark.

--enable-kerosin=no builds SimSpark without rendering support

--prefix=/some/path defines the path where the 'make install' will install the SimSpark executable, plugins and resources.

If omitted it defaults to /usr/local

\$ cd rcsoccersim/rcssserver3D/
\$ ./bootstrap
\$ ./configure
\$ make
\$ sudo make install

\item Make sure the linker can find your shared libraries if you changed
the install prefix as described above with the --prefix option.

\$ sudo gedit /etc/ld.so.conf
[add your install prefix if it isn't already there, save and close]
\$ sudo ldconfig

\item run the simulation
\$ cd
\$ simspark

\end{enumerate}

\section{How to Start a Simulation}

\section{How to Stop a Simulation}

\section{An Example of a Simulation Run}

%Take the user through the steps of running the soccer simulation and show what to expect (screen shots)

%I think for the initial outline of the manual it's easier to just focus on either app/simpark app/monitorspark  or rsgedit.

%%% Local Variables: 
%%% mode: latex
%%% TeX-master: "user-manual"
%%% End: 
