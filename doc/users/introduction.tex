\chapter{Introduction}

\section{What is Simspark?}

\section{History of the Project}

%Started out as rcssserver3D, first competition in Lisbon 2004, etc.
%Reference Marcos Thesis as the basis for the initial server development

On the RoboCup 2003 Symposium, a new approach to a three-dimensional
physically realistic soccer simulation was proposed. In a road map
discussion for the Soccer Simulation League on RoboCup 2003, the
participants decided on adding the three-dimensional simulation to the
competitions. In 2004 a simulation built on top of SimSpark was used
officially for the first three-dimensional RoboCup Simulation League
competition.

One of the long term goals of the soccer simulation is to aim for
realism. The long term objective are realistic humanoid players in a
physical environment that one day can challenge the champion of the
most recent World Cup. SimSparks two dimensional predecessor
simulation models the players and the ball as flat spheres.

It further lacks a realistic physical environment. The initial
SimSpark version of the soccer simulation took the next step into the
third dimension. This version modeled players as spheres in a physical
three dimensional world. Since then SimSpark grew even more feature
and now supports humanoid players with articulated bodies.

The soccer simulation was developed in parallel with the SimSpark
simulator. It served from the beginning as a testbed and a guide for
essential new features that were added to the simulator during
development. However changes to the simulator core were never
customized for the soccer simulation. Instead wed implemented generic
simulator services implemented and contained all soccer specific
details are contained in a set of plugins.

\section{About this Manual}

\section{Reader's Guide to the Manual}

%Purpose, contact persons, how to contribute

This manual describes the SimSpark simulator. Like the simulator
itself it is subject to constant change in an ongoing development
effort. It assumes that you are familiar with the basic concepts of
multiagent simulations.

It aims to be a guide on how to develop your own RoboCup agents,
construct new robot models, build your own custom monitor or use the
trainer command protocol to test your agents.

If there are errors, inconsistencies, or oddities, please notify
<EMail Address> with the location of the error and a suggestion of how
it should be corrected. We are always looking for anyone who has an
idea on how to improve the manual, as well as proofread or rewrite a
section of the manual.

The latest manual can be downloaded at <URL>.

%%% Local Variables: 
%%% mode: latex
%%% TeX-master: "user-manual"
%%% End: 
