
\chapter{Documentation with \LaTeX}
%\label{cha:DocumentationwithLatex} 


\section{Why \LaTeX}

\LaTeX\ is a computer program for typesetting documents.  It
reads a text file prepared according to the rules of
\LaTeX\, and converts it to other file formats's like DVI, PDF and so on.
The advantages of \LaTeX\ are
\begin{itemize}
\item It is platform independent
\item It is freeware
\item It can handle big documents with many pictures
\item There are many features to write down mathematical symbols and formulas
\item You can print the documents on a very high quality
\end{itemize}

\LaTeX\ (created by L. B. Lamport) is one of a number of `dialects' of \TeX, all based %%@
on the version of \TeX\ created by D. E. Knuth which is known as Plain \TeX.
It is particularly suited to the production of long articles and books, because it has
facilities for the automatic numbering of chapters, sections, theorems, equations etc., 
and also has facilities for cross-referencing.

The current version of \LaTeX\ is \LaTeX2e, this version was released in
1992. Since then it was updated twice a year.


\section{Installation}

\subsection{Linux}

\TeX\ and \LaTeX\ are part of nearly every Linux distribution.
No special installation or download is needed.

Although any text editor may be used to create \TeX\ 
you might want to use special purpose editors like the ones below:

\begin{itemize}
\item NEdit 5.3
\item Kile
\item LyX
\end{itemize}

\textbf{Kile} is part of most Linux distributions. 
It is available on http://kile.sourceforge.net.
Kile provides an editor with syntax highlighting, and many useful features, 
like creating DVI of PDF documents.
Also you can insert many \TeX\ commands by clicking a button, 
so you haven't to learn all commands.\\

\textbf{LyX} is a WYSIWYG editor for \LaTeX. 
Using LyX you can write a document in layout view.


\subsection{Windows}

You can also use \LaTeX\ under Windows. 
There are various \LaTeX\ distributions under Windows.

MiKTeX is a free distribution of \TeX\ and \LaTeX\ under Windows.
The SetupWizard of MikTeX can be downloaded from
 
http://prdownloads.sourceforge.net/miktex/setup.exe?download

After starting the setup wizard you have the choice of downloading the installation files 
or to install \TeX\ directly.

Again for editing \LaTeX\ every text editor is possible.
But there are some special text editors with syntax 
highlighting and buttons to create DVI and PDF documents.

Some editors under Windows are 

\begin{itemize}
\item WinTeX 2000, shareware available under www.tex-tools.de
\item WinEdt 5.3, shareware available under www.winedt.com
\end{itemize}

Both editors are shareware with trial versions for 30 days. 
They provide syntax highlighting and many more features.

\section{The first document}

The following document produces

\begin{verbatim}

\documentclass{article}

\begin{document}

Hello \LaTeX

\end{document}

\end{verbatim}

Hello \LaTeX\\



\LaTeX\ has many commands, to format text.
Commands always consist of three components:

\begin{verbatim}
\textsl{This will be printed in slanted font.}

The command  : \textsl
The braces   : { and }
The argument : The text inside the braces
\end{verbatim} 



\begin{verbatim}
\begin{document}
\end{document}
\end{verbatim}

A piece of text delimited by a named 
\begin{verbatim}
\begin - \end
\end{verbatim}
 pair ist called an environment.
Declarations written inside the begin-end pair affect all the text within that environment.

The \% symbol is the start of a comment.


\section{Important commands}

\subsection{Size of fonts}
The sizes of fonts used in \LaTeX\ are can be determined
and changed by means of the control sequences
\verb/\tiny/,
\verb/\scriptsize/,
\verb/\footnotesize/,
\verb/\small/,
\verb/\normalsize/,
\verb/\large/,
\verb/\Large/,
\verb/\LARGE/,
\verb/\huge/ and
\verb/\HUGE/:


{\tiny This text is \texttt{tiny}}.

{\scriptsize This text is \texttt{scriptsize}}.

{\footnotesize This text is \texttt{footnotesize}}.

{\small This text is \texttt{small}}.

{\normalsize This text is \texttt{normalsize}}.

{\large This text is \texttt{large}}.

{\Large This text is \texttt{Large}}.

{\LARGE This text is \texttt{LARGE}}.

{\huge This text is \texttt{huge}}.

{\Huge This text is \texttt{Huge}}.


\subsection{The fonts}

The \emph{shape} of a font can be \textup{upright},
\textit{italic}, \textsl{slanted} or \textsc{small caps}:
\begin{itemize}
\item
\textup{The LaTeX command}
   \verb/\textup{/\emph{text}\verb/}/
   \textup{typesets the specified text with an upright shape:
   this is normally the default shape.}
\item
\textit{The LaTeX command}
   \verb/\textit{/\emph{text}\verb/}/
   \textit{typesets the specified text with an italic shape.}
\item
\textsl{The LaTeX command}
   \verb/\textsl{/\emph{text}\verb/}/
   \textsl{typesets the specified text with a slanted shape:
   slanted text is similar to italic.}
\item
\textsc{The LaTeX command}
   \verb/\textsc{/\emph{text}\verb/}/
   \textsc{typesets the specified text with a small caps shape
   in which all letters are capitals (with uppercase letters taller than
   lowercase letters).}
\end{itemize}


\subsection{Format}

Two slash(\begin{verbatim} \\ \end{verbatim}) at the end of a line produce a new paragraph.

\begin{verbatim}
\begin{center}
This Text is centered
\end{center}
\end{verbatim}

produce 

\begin{center}
This Text is centered
\end{center}


\subsection{Producing Mathematical Formulae using \LaTeX}

Latex supports almost every math symbol and every possible formulae.

Some examples for the math symbols

... 


For example:

\begin{verbatim}
\[ ds^2 = dx_1^2 + dx_2^2 + dx_3^2 - c^2 dt^2 \]
\end{verbatim}

produce 

\[ ds^2 = dx_1^2 + dx_2^2 + dx_3^2 - c^2 dt^2 \]





\subsection{Tables}

TODO Christian: may be too detailed. Can be read in the internet

\begin{tabbing}
\textsc{NAME} \=col1 \=col2 \=col3 \=col4 \\
\>col1 \>col2 \>col3 \>col4 \\
\>col1 \>col2 \>col3 \>col4 \\
\>col1 \>col2 \>col3 \>col4 \\
\end{tabbing}


\begin{table}[htbp]
    \begin{tabular}[h]{|l|r|l|r|r|}
	  \hline
      \multicolumn{2}{|c|}{\textbf{Basic Parameters}} &
      \multicolumn{3}{|c|}{\textbf{Parameters for heterogeneous
        Players}} \\
      \multicolumn{2}{|c|}{\file{server.conf}} &
      \multicolumn{3}{|c|}{\file{player.conf}} \\ \hline
      \textbf{Name} & \textbf{Value} & \textbf{Name} & \textbf{Value} &
      \textbf{Range} \\ \hline
      \sparam{minpower} & -100 & & & \\ \hline
      \sparam{maxpower} & 100 & & & \\ \hline
      \sparam{stamina\_max} & 4000 &  & & \\ \hline
      \sparam{stamina\_inc\_max} & 45 &
      \sparam{stamina\_inc\_max\_delta\_factor} & -100.0 & \\
      & & \sparam{player\_speed\_max\_delta\_min} & 0.0 &
      \rb{1.5ex}{25} \\
      & & \sparam{player\_speed\_max\_delta\_max} & 0.2 &
      \rb{1.5ex}{---~45}\\\hline
      \sparam{extra\_stamina}$^a$ & 0.0 &
      \sparam{extra\_stamina\_delta\_min} & 0.0 & 0.0 \\
      & & \sparam{extra\_stamina\_delta\_max} & 100.0 & ---~100.0 \\\hline
      \sparam{dash\_power\_rate} & 0.006 &
      \sparam{dash\_power\_rate\_delta\_min} & 0.0 & 0.006 \\
      & & \sparam{dash\_power\_rate\_delta\_max} & 0.002 & ---~0.008
      \\ \hline
      \sparam{player\_rand} & 0.1 & & & \\\hline
      \sparam{wind\_force} & 0.0 & & & \\\hline
      \sparam{wind\_dir} & 0.0 & & & \\\hline
      \sparam{wind\_rand} & 0.0 & & & \\\hline
      \sparam{player\_decay} & 0.4 &
      \sparam{player\_decay\_delta\_min} & 0.0 & 0.4 \\
       & & \sparam{player\_decay\_delta\_max} & 0.2 & ---~0.6 \\ \hline
	\end{tabular}

\end{table}

  
	  


\section{Literatur and books}

At the WWW it gives a lot of tutorials and articals on \TeX and \LaTeX for additional information.




