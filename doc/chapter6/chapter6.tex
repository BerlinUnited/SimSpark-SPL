
\chapter{Documentation with \LaTeX}



\section{Why \LaTeX}

\LaTeX\ is a computer program for typesetting documents.  It
reads a text file prepared according to the rules of
\LaTeX\, and converts it to other file formats's like DVI, PDF and so on.
The advantages of \LaTeX\ are
\begin{itemize}
\item It is platform independent
\item It is freeware
\item It can handle big documents with many pictures
\item There are many features to write down mathematical symbols and formulas
\item You can print the documents on a very high quality
\item You can create tables and fill them
\end{itemize}

\LaTeX\ (created by L. B. Lamport) is one of a number of 'dialects' of \TeX, all based on the version of \TeX\ created by D. E. Knuth which is known as Plain \TeX.
\LaTeX\ extends \TeX\ of a various number of macros, which support much new functionality.
It is particularly suited to the production of long articles and books, because it has
facilities for the automatic numbering of chapters, sections, theorems, equations etc.,
and also has facilities for cross-referencing.

The current version of \LaTeX\ is \LaTeX2e, this version was released in 1992. Since then it was updated twice a year.

\newpage


\section{Installation}

\subsection{Linux}

\TeX\ and \LaTeX\ are part of nearly every Linux distribution.
No special installation or download is needed.

Although any text editor may be used to create \TeX\
you might want to use special purpose editors like the ones below:

\begin{itemize}
\item NEdit 5.3
\item Kile
\item LyX
\end{itemize}

\textbf{Kile} is part of most Linux distributions. It is available on http://kile.sourceforge.net.
Kile provides an editor with syntax highlighting, and many useful features, like creating DVI of PDF documents. Also you can insert many \TeX\ commands by clicking a button, so you haven't to learn all commands.

\textbf{LyX} is a WYSIWYG editor for \LaTeX.
Using LyX you can write a document in layout view. The work with LyX, may be a little strange first, but after a while it allows you writing documents in a very comfortable way.


\subsection{Windows}

You can also use \LaTeX\ under Windows.
There are various \LaTeX\ distributions under Windows.

MiKTeX is a free distribution of \TeX\ and \LaTeX\ under Windows.
The SetupWizard of MikTeX can be downloaded from

http://prdownloads.sourceforge.net/miktex/setup.exe?download

After starting the setup wizard you have the choice of downloading the installation files
or to install \TeX\ directly.

\medskip

Again for editing \LaTeX\ every text editor is possible.
But there are some special text editors with syntax
highlighting and buttons to create DVI and PDF documents.

Some editors under Windows are

\begin{itemize}
\item WinTeX 2000, shareware available under www.tex-tools.de
\item WinEdt 5.3, shareware available under www.winedt.com
\end{itemize}

Both editors are shareware with trial versions for 30 days.
They provide syntax highlighting and many more features.

\newpage

\section{The first document}

The following document

\begin{verbatim}

\documentclass{article}

\begin{document}

Hello \LaTeX

\end{document}

\end{verbatim}

produces\\

Hello \LaTeX\\

\medskip

\LaTeX\ has many commands, to format text. Commands always consist of three components:

\begin{verbatim}
\textsl{This will be printed in slanted font.}

The command  : \textsl
The braces   : { and }
The argument : The text inside the braces
\end{verbatim}


\begin{verbatim}
\begin{document}
\end{document}
\end{verbatim}

A piece of text delimited by a named
\begin{verbatim}
\begin - \end
\end{verbatim}
 pair is called an environment.
Declarations written inside the begin-end pair affect all the text within that environment.\\

The \% symbol is the start symbol of a comment, the remain line will be comment.

\newpage

\section{Important commands}

\subsection{Size of fonts}
The sizes of fonts used in \LaTeX\ can be determined and changed by means of the control sequences

\verb/\tiny/,
\verb/\scriptsize/,
\verb/\footnotesize/,
\verb/\small/,
\verb/\normalsize/,
\verb/\large/,
\verb/\Large/,
\verb/\LARGE/,
\verb/\huge/ and
\verb/\HUGE/:

\medskip

{\tiny This text is \texttt{tiny}}.

{\scriptsize This text is \texttt{scriptsize}}.

{\footnotesize This text is \texttt{footnotesize}}.

{\small This text is \texttt{small}}.

{\normalsize This text is \texttt{normalsize}}.

{\large This text is \texttt{large}}.

{\Large This text is \texttt{Large}}.

{\LARGE This text is \texttt{LARGE}}.

{\huge This text is \texttt{huge}}.

{\Huge This text is \texttt{Huge}}.


\subsection{The fonts}

The \emph{shape} of a font can be \textup{upright},
\textit{italic}, \textsl{slanted} or \textsc{small caps}:
\begin{itemize}
\item
\textup{The LaTeX command}
   \verb/\textup{/\emph{text}\verb/}/
   \textup{typesets the specified text with an upright shape:
   this is normally the default shape.}
\item
\textit{The LaTeX command}
   \verb/\textit{/\emph{text}\verb/}/
   \textit{typesets the specified text with an italic shape.}
\item
\textsl{The LaTeX command}
   \verb/\textsl{/\emph{text}\verb/}/
   \textsl{typesets the specified text with a slanted shape:
   slanted text is similar to italic.}
\item
\textsc{The LaTeX command}
   \verb/\textsc{/\emph{text}\verb/}/
   \textsc{typesets the specified text with a small caps shape
   in which all letters are capitals (with uppercase letters taller than
   lowercase letters).}
\end{itemize}


\pagebreak
\subsection{Format}

A double backslash(\textbackslash \textbackslash) is used at the end of a line to force the program to start a new line.\\
\medskip

If you need some vertikal skip between to paragraphs you can write the commands: \\
\textbackslash smallskip = procude a small skip \\
\textbackslash medskip = produce a medium skip \\
\textbackslash bigskip = produce a big skip (the height of one line) \\

You can also use the \textbf{vspace} command:

\textbackslash vspace\{5mm\} produce a 5 mm vertical skip\\
\textbackslash vspace\{5pt\} produce a 5 pt vertical skip\\

\medskip


For space between words, there is the \textbf{hspace} command

\textbackslash hspace\{.25in\} produce a space of 1/4 inch\\
\textbackslash hspace\{-.25in\} produce this: \\
WORD1 \hspace{-0.25in} WORD2

\medskip

\textbf{Center environment}
\begin{verbatim}
\begin{center}
This Text is centered
\end{center}
\end{verbatim}

produce

\begin{center}
This Text is centered
\end{center}


\pagebreak
\subsection{Producing Mathematical Formulae using \LaTeX}

Latex supports almost every math symbol and every possible formulae.
But first you have to enter the mathematic mode. You can do that with an \$ symbol at the begin and the end of a command. Some examples for the math symbols\\

\bigskip

\textbf{The greek letters}\\
The \textbackslash alpha command produce the $\alpha$ letters. This is possible for every greek letter. If you want the upper case letter, just make the first letter upper case. Here are some examples:

\begin{itemize}
\item $\gamma$ \hspace{10pt} \textbackslash gamma
\item $\delta$ \hspace{10pt} \textbackslash delta
\item $\pi$ \hspace{10pt} \textbackslash pi
\item $\Pi$ \hspace{10pt} \textbackslash Pi
\item $\Phi$ \hspace{10pt} \textbackslash phi
\end{itemize}

\bigskip

\textbf{Mathematical symbols}\\
There are numerous mathematical symbols that can be used. Here are some examples:

\begin{itemize}
\item $\forall$ \hspace{10pt} \textbackslash forall
\item $\neg$ \hspace{10pt} \textbackslash neg
\item $\infty$ \hspace{10pt} \textbackslash infty
\item $\sum$ \hspace{10pt} \textbackslash sum
\item $\int$ \hspace{10pt} \textbackslash int
\item $\ne$ \hspace{10pt} \textbackslash ne
\item $\le$ \hspace{10pt} \textbackslash le
\item $\ge$ \hspace{10pt} \textbackslash ge
\end{itemize}

\bigskip

\textbf{Mathematical Formulae}
\LaTeX\ allows you to write very difficult formulae in a simple way. For example:

\begin{verbatim}
\[ ds^2 = dx_1^2 + dx_2^2 + dx_3^2 - c^2 dt^2 \]
\end{verbatim}

produce

\[ ds^2 = dx_1^2 + dx_2^2 + dx_3^2 - c^2 dt^2 \]



\pagebreak
\subsection{Tables}

You can also create tables in \LaTeX\. Here is a small example for a table, for additional information, please search the web.

\medskip

Here is a small example of a tabbing:

\begin{tabbing}
\textsc{NAME} \=col1 \=col2 \=col3 \=col4 \\
\>col1 \>col2 \>col3 \>col4 \\
\>col1 \>col2 \>col3 \>col4 \\
\>col1 \>col2 \>col3 \>col4 \\
\end{tabbing}

\medskip
Here is a small example of a table:

\begin{table}[htbp]
    \begin{tabular}[h]{|l|r|l|r|r|}
      \hline
      \multicolumn{2}{|c|}{\textbf{Basic Parameters}} &
      \multicolumn{3}{|c|}{\textbf{Parameters for heterogeneous
        Players}} \\
      \multicolumn{2}{|c|}{\file{server.conf}} &
      \multicolumn{3}{|c|}{\file{player.conf}} \\ \hline
      \textbf{Name} & \textbf{Value} & \textbf{Name} & \textbf{Value} &
      \textbf{Range} \\
      \hline
    \end{tabular}
\end{table}




\section{Structure of the Soccerserver documention}

You can find the documentation in the \"doc\" subdirectory in the soccerserver directory. There is a file named manual.tex which allows you to build the whole documention. The single chapters are placed in subdirectories which are named with the chapter number. For example chapter2, chapter3 and so on. You can only build the whole documention with all chapters and not a single chapter. If you want to build a single chapter you have to edit the chapter.tex file and insert a \textbackslash begin\{document\} at the beginning and a \textbackslash end\{document\} at the end of the file.\\



\section{Literature}

At the WWW it gives a lot of tutorials and articles on \TeX\ and
\LaTeX\ for additional information.

\smallskip

\textbf{Websites}
\begin{itemize}
\item \textit{www.latex-project.org/} \hspace{10pt} The \LaTeX\
project homepage

\item \textit{http://www.tug.org/} \hspace{10pt} \TeX\ Users Group
Web Site

\item \textit{http://www.ctan.org/} \hspace{10pt} CTAN: the
Comprehensive TeX Archive Network \TeX\ and \LaTeX\ and extensible
tools can be downloaded

\item \textit{http://www.miktex.org/} \hspace{10pt} Download a
Mik\TeX\ version

\end{itemize}

\smallskip


\textbf{Books}
\begin{itemize}

\item Donald E. Knuth: The \TeX Book,  Addison-Wesley Publishing
Company

\item Bernice Sacks Lipkin: \LaTeX\ for Linux

\item Michel Goosens, Frank Mittelbach and Alexander Samarin: The
\LaTeX\ Companion, Addison-Wesley

\end{itemize}
