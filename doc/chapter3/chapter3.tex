
\chapter{Working with CVS}

\section{About CVS}
CVS (Concurrent Versions System) is a tool used by many software developers to manage 
changes within their source code tree. CVS provides the means to store not only the current 
version of a piece of source code, but a record of all changes (and who made those changes) 
that have occurred to that source code. Use of CVS is particularly common on projects with 
multiple developers, since CVS ensures changes made by one developer are not accidentally 
removed when another developer posts their changes to the source tree.

Information about accessing this CVS repository may be found in our document titled, 
"Basic Introduction to CVS and SourceForge.net (SF.net) Project CVS Services".

In order to access a CVS repository, you must install a special piece of software called a 
CVS client; CVS clients are available for most any operating system. 
Information about CVS client software may be found in our document titled, 
"Basic Introduction to CVS and SourceForge.net (SF.net) Project CVS Services".

\section{Access to CVS}

\subsection{Anonymous CVS Access}
This project's SourceForge.net CVS repository can be checked out through anonymous (pserver) 
CVS with the following instruction set. The module you wish to check out must be specified as 
the modulename. When prompted for a password for anonymous, simply press the Enter key. 
To determine the names of the modules created by this project, you may examine their CVS 
repository via the provided web-based CVS repository viewer.

\begin{verbatim}
cvs -d:pserver:anonymous@cvs.sourceforge.net:/cvsroot/sserver login
cvs -z3 -d:pserver:anonymous@cvs.sourceforge.net:/cvsroot/sserver co modulename
\end{verbatim}

Information about accessing this CVS repository may be found in our document titled, 
"Basic Introduction to CVS and SourceForge.net (SF.net) Project CVS Services".

Updates from within the module's directory do not need the -d parameter.

NOTE: UNIX file and directory names are case sensitive. 
The path to the project CVSROOT must be specified using lowercase characters (i.e. /cvsroot/sserver)


\subsection{Developer CVS Access via SSH}

Only project developers can access the CVS tree via this method. 
A SSH client must be installed on your client machine. Substitute modulename 
and developername with the proper values. Enter your site password when prompted.

A significant amount of information about project CVS services may be found in our 
"Introduction to Project CVS Services". Developers new to CVS should read our 
"Basic Introduction to CVS and SourceForge.net (SF.net) Project CVS Services".

Developers may also make use of shared SSH keys for authentication.

\begin{verbatim}
export CVS_RSH=ssh
cvs -z3 -d:ext:developername@cvs.sourceforge.net:/cvsroot/sserver co modulename
\end{verbatim}

NOTE: UNIX file and directory names are case sensitive. 
The path to the project CVSROOT must be specified using lowercase characters (i.e. /cvsroot/sserver)

\subsection{Accessing a Branch}
\section{Writing to CVS}
\subsection{Checking in a File}
\subsection{Checking in from a Branch}
\subsection{Creating Tags}

