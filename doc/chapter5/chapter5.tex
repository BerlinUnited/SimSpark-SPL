
\chapter{CVS, Bug Tracking and Coding Standards}

%\label{cha:CVS,BugTrackingandCodingStandards}

\section{CVS}

This CVS-documentation is only a departure of the functionality of CVS. It is based on the CVS-documentation from the SourceForge.net website (www.sourceforge.net) and it describes how to use CVS with with the soccerserver project on SourceForge.net. For more detailed information please visit the SourceForge.net web-site.

\subsection{What is CVS?}

CVS, Concurrent Versions System, is a tool used by many software development teams to store their source code in a centralized location. CVS allows everyone to obtain a copy of this source code (i.e. everyone is provided read-only access) and allows developers of the software to also make changes to the source code stored in that central location.\\

Many development teams use CVS to manage their source code because CVS performs two very helpful jobs. First, CVS tracks each of the changes made by developers; including what was changed, when it was changed, and who made the change. Second, CVS ensures that the changes made by each developer on a development team do not accidentally overwrite changes made by other developers.

\subsection{Be careful}

If you are using a Windows CVS-client you must be careful if you want to transfere data or files to the soccerserver project. Because all files there are in a Linux format. But a Windows CVS-client uploads this files in a Windows format. This is critical because Windows uses an other sign for the linefeed then Linux. This will result in not beeing able to run for example the configure scripts. And this would result in great confusion.
But almost all Windows CVS-clients have the option to transfere the data in a Linux mode which solve that problem. Please be carefull with that!

\subsection{Getting started}

\subsubsection{What's to do?}

To establish a connection to the soccerserver project on SourceForge.net you have to install and configure following programs:

\begin{itemize}
\item a Linux-system
\item a CVS-client (see chapter "Some CVS-programs")
\item a SSH-client (only if you want to edit the soccerserver repository (most users want to))
\end{itemize}

The following chapters will describe how to configure the programs to be ready to work on the soccerserver-project. 

\subsubsection{Setting the pathes in the CVS-client}

Especially in GUI-based CVS-clients you have to set several pathes (in console based CVS-clients you tipe them directly in for almost each command). For the soccerserver project they are:

\begin{itemize}
\item for an anonymous-connection:\\
authentication:	pserver\\
Path: /cvsroot/sserver\\
Host address: cvs.sserver.sourceforge.net\\
User name: anonymous\\
CVSROOT: :ext:anonymous@cvs.sserver.sourceforge.net:/cvsroot/sserver\\

\item for a SSH-connection:\\
authentication: ssh\\
Path: /cvsroot/sserver\\
Host address: cvs.sserver.sourceforge.net\\
User name: your username\\
CVSROOT: username@cvs.sserver.sourceforge.net:/cvsroot/sserver\\

\end{itemize}

\subsubsection{Uploading your SSH-key}

If want to connect with SSH to SourceForge.net you first have to generate a public key with your SSH-clent. Detailed information in how to generate a public key you'll find in the documentation of your SSH-client.
To upload your public (not private!!!) key to SourceForge.net you have to

\begin{enumerate}
\item Go to your user account web-site on SourceForge.net (https://sourceforge.net/account/) and login.
\item Follow the link "Edit Keys".
\item Open the file containing your public key generated by your SSH-client in an editor like NotePad (Windows) or Kate (Linux).
\item Copy the whole content to the clipboard.
\item Within your web browser, paste the data in to the provided text box, then click the 'Update' button.
\item Key data will be synchronized from the SourceForge.net web site to the project shell and project CVS servers every six hours. So please wait six hours with establishing a connection to SourceForge.net.
\end{enumerate} 

\subsection{Some CVS-programs}

\begin{itemize}
\item For Microsoft Windows platforms: TortoiseCVS (http://tortoisecvs.sourceforge.net/)
\item For Microsoft Windows platforms: WinCvs (http://cvsgui.sourceforge.net/)
\item For Microsoft Windows platforms: Cygwin (http://cygwin.com/)
\item For Linux, BSD and Mac OS X platforms: CVS (normally provided by your Operating System vendor (http://cvshome.org/)
\item For Mac OS X platforms: CVS provided by fink (fink includes a large number of UNIX-style tools not included in the standard Mac OS X distribution; fink includes a more recent version of CVS than included with Mac OS X)(http://fink.sourceforge.net/)
\item For Mac OS classic: maccvs (http://cvsgui.sourceforge.net/)
\item For Mac OS classic: MacCVS Pro (http://sourceforge.net/projects/maccvspro/)
\end{itemize}

If you are a project developer and will be writing to the CVS repository, you will require an additional piece of software, an SSH client. Additional information regarding these requirements may be found in our "Introduction to SourceForge.net Project CVS Services for Developers" (https://sourceforge.net/docman/display\_doc.php?docid=768\&group\_id=1\#top)\\

Please note that SourceForge.net does provide step-by-step instructions for developers to use in installing their CVS client and configuring it to work with SSH. The documents containing these instructions may be found in the SourceForge.net Site Documentation (https://sourceforge.net/docman/index.php?group\_id=1). This level of documentation is not provided for non-developers, since configuring your client for pserver-based access (read-only) is comparatively easy; all CVS clients work with pserver-based access immediately after they have been installed.\\

\subsection{Basic CVS terminology}

CVS client - You may wish to think of CVS as a remote server which contains a set of files. A special protocol is used to get data from a CVS server; this protocol is not the same as FTP nor as the protocols that your web browser uses. In order to retrieve those files from the CVS server, you will need a special piece of software called a CVS client which knows how to talk to the CVS server.

CVS repository - The CVS server stores a copy of the software and data that the project has uploaded to that server. The server retains both the most recent version of each file and every historical version of the file (past changes). This copy of the software and data uploaded by the project is called a CVS repository. Each project hosted on SourceForge.net has its own repository.

module - CVS provides a way to store different types of data within the same repository. Each distinct set of data a project stores within their repository (i.e. source code for a program they are developing, source code for a library they are developing, and their project web site) will be placed within its own module. When you retrieve data from the project CVS repository, you will need to know the name the project used for the module that contains the data you want. Modules contain files and directories, just like the filesystem on your computer does. Within our web-based CVS repository viewer, modules are shown as top-level directories within the repository.

working copy - Though the CVS repository stores every version of every file that has been uploaded to the repository, when you retrieve data from the CVS repository using your CVS client, only one version of each file is saved to your hard drive. The copy of the data you get from the CVS server is called a "working copy". A working copy is obtained using the "checkout" command. When you checkout your working copy, you will specify which module you wish to retrieve; only the files and directories located in that module will be placed in your working copy.

commit - When a developer makes a set of changes to the files in their working copy, they eventually need to upload those changes back to the CVS server. This process of uploading updated files to the CVS server is called a "commit". When a developer performs a CVS commit, only the changed files are sent to the server (which saves bandwidth and time). Only developers will commit changes to a CVS repository.

update - When you checkout your working copy of the repository, it will contain the latest versions of each file. Development typically continues, however, and your working copy will be out-of-date after the next commit is performed by a developer. To bring your working copy back up-to-date, you will perform an update using the "update" command. By performing an update, all of the changes are merged in to your working copy; performing an update saves time over the alternative, removing your working copy and performing a fresh checkout.

\subsection{The most important CVS-commands}

\subsubsection{Login to a CVS repository (login)}

If you are making use of anonymous pserver-based CVS repository access (i.e. you are not a developer with write access), you will login to the CVS repository before you perform other commands. The password used for anonymous login to all SourceForge.net repositories is empty (hit enter at the password prompt).

Developers (who use SSH for authentication) will not use this command (their login occurs before each command is performed, rather than being a separate step). This is true even if you are using a graphical CVS client (like WinCvs) -- developers using SSH authentication will not be able to use the "login" command, since it would serve no functional purpose.

\subsubsection{Creating a new module (import)}

In order to get an initial set of data in to the CVS repository, a developer will perform an "import". The import operation will create the module and populate it with the data in the current directory (when you perform the import command, you should be in the directory that contains the exact materials you want to have inside the module, not one directory up from that point).

\subsubsection{Retrieving the repository (checkout)}

Once a project has placed data in their CVS repository, you may checkout a working copy of that data. In order to checkout a working copy, you must know the name of the module (modules are top level directories in the repository) that they have used to store the data you wish to retrieve. We recommend using our web-based CVS repository access to see the list of the modules available in the repository. In graphical CVS-clients you dont't need to use the web-based CVS repository. There you can retrieve the module-names with an build-in feature.

\subsubsection{Uploading changed files (commit)}

When a developer has made changes within their working copy and wishes to upload the changed files back to the CVS repository, they perform a commit. With each commit, you also include a commit message, describing the changes made to the files you are committing (i.e. "Fixes bug reported on Support Request 627956").

\subsubsection{Uploading new files (add)}

After the initial import has occurred, developers will use the "add" command when they wish to add a file or a directory to the CVS repository. When operating against a file, the "add" command schedules the creation of the file within the repository and the actual creation occurs when a "commit" is performed. When operating against a directory, the "add" command creates the directory within the repository immediately. In order to add a directory and its contents to the repository, both the directory and the files must be added (adding the directory does not automatically add the files). Before performing the "add", you must first create the files and directories to be added within your working copy.

\subsubsection{Removing a file from the CVS-server (remove)}

When you wish to remove a file from the repository, you will first remove the file from your working copy, then you will use the "remove" command to schedule the removal of the file. When a file is removed, CVS retains a copy of that file within a directory named "Attic", located in the CVS repository. The "remove" command schedules the removal of the file; when the next commit occurs, the file is moved in to the Attic subdirectory (of the directory where the file original resided).

CVS retains files removed with the "remove" command because they may be needed either for historical record (CVS tries very hard to keep historical information about everything that occurs in your repository) or because the file may exist in another branch within your repository (branching is a more advanced topic, covered in the CVS book and CVS manual, which we will overlook in this introduction).

In order to actually purge a file from the repository (i.e. if it contains sensitive information, or if you are performing significant restructuring), you will need to contact the SourceForge.net team. Additional information regarding the variety of assistance the SourceForge.net team can provide in managing your repository may be found in our "Introduction to SourceForge.net Project CVS Services for Developers".

\subsubsection{Keeping your downloaded working copy up-to-date (update)}

After you have checked out a working copy from a repository, you will keep that working copy up-to-date (i.e. ensure that your working copy includes changes committed by (other) developers on the project) through the use of the "update" command. Almost everyone will want to use the -d (build directories if new directories have been added to the repository since your checkout or last update) and -P (leave empty directories out of your working copy) options for the update command.

\subsection{Interest web-sites to visit}

\begin{itemize}
\item One of the finest CVS learning resources available is The CVS Book, "Open Source Development with CVS" (by Karl Fogel). The technical portions of this book are available freely at: http://cvsbook.red-bean.com
\item The CVS manual is also an excellent reference, though most users find the CVS Book to be easier to understand (http://www.cvshome.org/docs/manual/cvs-1.11.6/cvs.html).
\item CVS clients often include documentation in the form of online help, a written manual included with their file releases, or web-based documentation on their web site. These are all excellent sources for information on how a particular CVS client works. All CVS clients offer essentially the same features, but each have a different menu layout (if they have a graphical front-end); the documentation offered with the CVS client is an excellent way to acclimate yourself with the way the client works. One example (helpful for WinCvs users) is the WinCvs Daily Use Guide (http://cvsgui.sourceforge.net/howto/index.html). 
\end{itemize}

\section{Bug Tracking}

\subsection{Purpose of the Bug Tracking System}

The Bug Tracking System provides a platform for reporting newly found bugs or to
browse those reports. In these reports is a detailed description of the bug, the author of the bug report and the responsible programmer.

\subsection{How to access the Bug Tracking System?}

\subsubsection{How to find it?}

The website of the Bug Tracking System can be found on The Robocup Soccer Simulater website on sourcefourge.net (http://sserver.sourceforge.net/). Follow the links "Tracker" or "Bugs".

\subsubsection{Usage of the Bug Tracking System}

If you are not logged in or you have no account to the Robocup Soccer Simulater project
you have only two options, "Browse" and "Admin". The link "Admin" is only for administrators of the
Robocup Soccer Simularer project. So you only can browse the bug reports.

If you are logged in you have the extra options, "Submit New" and "Reporting".
Following the link "Submit New" your browser loads a form for reporting a new bug.
After following the link "Reporting" you are able to generate several reports about
bugs, feature requests, new user request, patches, support requests and void reports.

\section{Coding Standarts}

This coding standards were defined by Marco K\"{o}gler from Universit\"{a}t Koblenz.\\

This is the coding style used for the initial parts of the 3D simulator. PLEASE everyone contributing try to use it as well. Even in case you don't like specific parts: changes in the style guide would mean either inconsistencies or cause lot of work for changing the already existing code. 

\subsection{Tabs}

Never use hard tabs in your source files (have your editor insert spaces instead of tabs). This guideline is especially important. The use of hard tab characters ends up causing problems because not every editor uses the same number of spaces per tab character (in fact, most editors allow you to configure how many spaces there are per tab character). If you open a file with hard tabs already in it, instead of modifying the code so that it is readable in your own editor because your tab stops are different, please remove the tabs. There are several utilities which will do this. See also http://www.jwz.org/doc/tabs-vs-spaces.html.

\subsection{Indentation}

Only four-space line indentation should be used. 

\subsection{Placing Braces}

Opening and closing braces should be on the same level of indentation below the token opening the scope. Opening and closing braces appear on a separate line by themselves.
\\

Example:
\begin{verbatim}

void
ClassName::function()
{
    if (x)
    {
        while (y)
        {
            // ...
        }
    }
    else
    {
        switch (z)
        {
            case 1:
            {
                // ...
                break;
            }
            default:
            {
                // ...
                break;
            }
        }
    }
}

\end{verbatim}
\subsection{Naming Conventions}

\begin{verbatim}

class         ClassName;    // not Class_Name 
struct        StructName;   // like classes
typedef int   TMyInt;       // start with a capital T, 
                            // followed by a capital letter
union		UnionName;    

\end{verbatim}

\subsubsection{Enums}

Enumeration types 'enum' are starting with 'E', followed by a type-name starting with a capital letter.

Enumeration elements start with 'e', followed by the name of the constant starting with a capital letter.

Example:
enum EEnumName
{
    eEnumElement1,
    eEnumElement2
};

\subsubsection{Functions}
\begin{verbatim}

void 
ClassName::FunctionName()

- Function name starting with capital letter (!)
- optional: place the return type on a separate line 
- no blanks between function name and parenthesis

\end{verbatim}
\subsubsection{Variables}

*** Variables with local scope 

Variables with local scope start with a small letter. New words
within the variable start with capital letters.

Example: myVariable
\\

*** Globally defined variables and constants

Global constants and variables start witg 'g', followed by the
variable name starting with a capital letter. New words within the
variable name start with capital letters.
\\

*** Member variables

Member variables start with 'm', followed by the variable name
starting with a capital letter. New words within the variable name
start with capital letters.
\\

Example: mMemberVariable
\\

Constant static member variables are the same as the
non-static types. 
\\

Examples: mStaticConstMemberVariable

\subsubsection{Macros}
\begin{verbatim}

Macro names consist of only capital letters. Multi word macros are
separated by underscore '_'.

Example: MY_MACRO

\end{verbatim}
\subsection{Comments}
\begin{verbatim}

Please use comments suitable for doxygen to get a nice HTML
documentation. Comments should be placed above the respective
elements

Example:

/*! This is a comment 
    for more than one line
*/

//! This is a comment for exactly one line

\end{verbatim}
\subsection{Gotchas}
\begin{verbatim}

There are some standard comments for parts of the code ('gotchas'): 
':TODO:',':BUG:'. Gotchas are the first part of a comment.

Example:

//:BUG: There is a bug below someone has to remove

/*:TODO: The stuff below
        needs some clean up
*/

\end{verbatim}
\subsection{Files}
\begin{verbatim}

- file names consist of only small letters
- .h is the suffix for c++ header files, .cpp is the suffix for c++ source 
  files
- max line length is 80 characters
- include copyright:

/* -*- mode: c++; c-basic-indent: 4; indent-tabs-mode: nil -*-
   
this file is part of rcssserver3D
Fri May 9 2003
Copyright (C) 2002,2003 Koblenz University
Copyright (C) 2003 RoboCup Soccer Server 3D Maintenance Group
$Id: chapter5.tex,v 1.4 2004/01/07 16:20:46 gato_negro Exp $

This program is free software; you can redistribute it and/or modify
it under the terms of the GNU General Public License as published by
the Free Software Foundation; version 2 of the License.
  
This program is distributed in the hope that it will be useful,
but WITHOUT ANY WARRANTY; without even the implied warranty of
MERCHANTABILITY or FITNESS FOR A PARTICULAR PURPOSE.  See the
GNU General Public License for more details.
 
You should have received a copy of the GNU General Public License
along with this program; if not, write to the Free Software
Foundation, Inc., 675 Mass Ave, Cambridge, MA 02139, USA.
*/

\end{verbatim}
\subsubsection{Header files}
\begin{verbatim}

Header files are protected by include guards, and the autoconf config.h file
should be included:

#ifndef MY_HEADER_FILE_H
#define MY_HEADER_FILE_H

#ifdef HAVE_CONFIG_H
#include <config.h>
#endif

But this usually doesn't take namespaces into account, which means that 
sometimes header files from different libs can clash, so I've started using

\#ifndef NAMESPACE_MY_HEADER_FILE_H
\#define NAMESPACE_MY_HEADER_FILE_H

\#ifdef HAVE_CONFIG_H
\#include <config.h>
\#endif

\end{verbatim}

where namespaces are used (which should be most of the time).

\subsubsection{Class declaration style}
\begin{verbatim}

class ClassName
{
public:
	// public types

protected:
	// protected types

private:
	// private types
	
public:
	// public functions

protected:
	// protected functions

private:
	// private functions

public:
	// public members

protected:
	// protected members

private:
	// private members
};

\end{verbatim}
\subsection{Options for GNU indent to fix broken code}
\begin{verbatim}

These are the options for GNU indent to get your code looking
pretty much OK. Since indent seems to be a little bit non-deterministic, 
better don't use it for files that are looking almost perfect :). 

-nbad -bap -sob --dont-star-comments --no-comment-delimiters-on-blank-lines
-cp33 -d0 -nfc1 -nfca -bl -bli0 -cli4 -cbi0 -ss -npcs -nprs -saf -saw -sai -cs
-ip0 -di1 -nbc -psl -bls -i4 -nut -ci0 -di1 -lp -l80 -nbbo -hnl

\end{verbatim}

