\chapter{Unit Tests}

Unit tests help to ensure that code units work according to their specification.
They help to increase the code quality in two ways: First they directly 
ensure that code covered by test cases contains fewer bugs. Second they indirectly
improve the code quality by increasing a developers confidence that changes
to a piece of code did not introduce bugs. Therefore the barrier for making
necessary design changes is much lower for code covered by unit tests.
 
CppUnit is the C++ port of the famous JUnit framework for unit testing. 
it simplifies the creation and execution of unit tests for C++ code.
Test output is in XML or text format for automatic testing and GUI based for
supervised tests.

\section{Installation}

\subsection{Downloads}
Necessary files for CPPUnit can all be downloaded from the CPPUnit page:
\begin{verbatim}
http://sourceforge.net/project/showfiles.php?group_id=11795
\end{verbatim}

By now the latest version of CPPUnit is 1.8.0

For the first steps you only have to download the following files:
\begin{itemize}
\item cppunit-1.8.0.tar.gz
\item cppunit-docs-1.8.0.tar.gz
\end{itemize}

\subsection{Linux}

TODO Sebastian:

\subsection{Windows}

Currently, the only supported WIN32 platform is Microsoft Visual C++ version 6.0 or later.
We tried with MS Visual C++ Standard Edition and it worked.

The libraries and DLL can all be built from the
\textit{src/CppUnitLibraries.dsw} workspace which is located in your CPPUnit directory.

For additional information read the 
\textit{INSTALL-WIN32.TXT} located in the CPPUnit directory. There you also have 
a list of the different libraries of which you can choose your
preferred library.

\subsection{Documentation}
There are two possibilities to access the documentation of the CPPUnit library: 
The documentation can be integrated in the MSDN library or simply can be read as 
HTML version.

To use the HTML documentaion extract the archive cppunit-docs-1.8.0.tar.gz to 
the desired directory and start with the file index.html in any browser you prefer.

If you want to integrate the documentation in the MSDN library you have to 
download \begin{verbatim} CppUnitMsdnDoc-1_8_0.zip \end{verbatim}
Have a look at \textit{Installing-Doc.txt} and follow the steps described there.


\section{Creating Unit Tests}

To write your first unit tests follow the instructions in the
\begin{verbatim} CPPUnit Cookbook \end{verbatim} which you can find on the first page of
the HTML documentation mentioned above.

Another way to learn how to make unit tests is, to look at any existing example 
of a unit test and try to understand what is done there.

To write your first unit tests with MS Visual C++, you have to do the following steps:
\begin{itemize}
\item RTTI must be enabled (Project : Settings : C++ : C++ Language)
\item link against \textit{lib/cppunitXX.lib} (XX for the chosen library)
\item add the include directory of your cppunit directory to the search path
\end{itemize}

TODO Sebastian: example

