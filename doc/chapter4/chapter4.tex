\chapter{Unit Tests}

Unit tests help to ensure that code units work according to their specification.
They help to increase the code quality in two ways: First they directly
ensure that code covered by test cases contains fewer bugs. Second they indirectly
improve the code quality by increasing a developers confidence that changes
to a piece of code did not introduce bugs. Therefore the barrier for making
necessary design changes is much lower for code covered by unit tests.

CPPUnit is the C++ port of the famous JUnit framework for unit
testing. It simplifies the creation and execution of unit tests
for C++ code. Test output is in XML or text format for automatic
testing and GUI based for supervised tests.

\section{Why should unit tests be used?}

There are different issues why unit tests are a good solution to
avoid mistakes and save time during the software development.

Often developers think that testing their code would take too much
time, because they are under pressure. But with that, they go
wrong. With unit tests many faults and their programm wide effects
can be avoided or detected early, so that intensive debug sessions
become no longer necessary. The result could be an enormous saving
of time.


\section{Installation}

\subsection{Downloads}
Necessary files for CPPUnit can all be downloaded from the CPPUnit page:
\begin{verbatim}
http://sourceforge.net/project/showfiles.php?group_id=11795
\end{verbatim}

By now the latest version of CPPUnit is 1.8.0

For the first steps you only have to download the following files:
\begin{itemize}
\item cppunit-1.8.0.tar.gz
\item cppunit-docs-1.8.0.tar.gz
\end{itemize}

\subsection{Linux}

A very detailed instruction on howto install CPPUnit on a
linux/unix machine is located in the two files of the downloaded
package:
\begin{itemize}
\item INSTALL
\item INSTALL-unix
\end{itemize}


To check if CPPUnit the CPPUnit libraries and include files are
installed or accessible in your linux/unix machine you can type at
the prompt
\begin{center}
\$ cppunit-config [--version] [-cflags] [--libs] [--prefix]
[--help]
\end{center}
Using the different optional flags you can inquire the system
about the CPPUnit installed version, the cflags and library path
you have to use, where CPPUnit was installed and this help.

\subsection{Windows}

Currently, the only supported WIN32 platform is Microsoft Visual
C++ version 6.0 or later, but as CPPUnit uses ANSI C++, there are
few ports to other environments like C++Builder. We tried with MS
Visual C++ Standard Edition and it worked.

The libraries and DLL can all be built from the
\textit{src/CppUnitLibraries.dsw} workspace which is located in your CPPUnit directory.

For additional information read the
\textit{INSTALL-WIN32.TXT} located in the CPPUnit directory. There you also have
a list of the different libraries of which you can choose your
preferred library.

\subsection{Documentation}
There are two possibilities to access the documentation of the CPPUnit library:
The documentation can be integrated in the MSDN library or simply can be read as
HTML version.

To use the HTML documentaion extract the archive cppunit-docs-1.8.0.tar.gz to
the desired directory and start with the file index.html in any browser you prefer.

If you want to integrate the documentation in the MSDN library you have to
download \begin{verbatim} CppUnitMsdnDoc-1_8_0.zip \end{verbatim}
Have a look at \textit{Installing-Doc.txt} and follow the steps described there.


\section{Creating Unit Tests}

To write your first unit tests follow the instructions in the
\textbf{CPPUnit Cookbook} which you can find on the first page of
the HTML documentation mentioned above. The CppUnit Cookbook gives
you a first idea of what is a "Fixture" , a "Test Case", a "Test
Suite", the "TestRunner" and the "TestFactoryRegistry".

Another way to learn how to make unit tests is, to look at any
existing example of a unit test and try to understand what is done
there. A package of examples can be downloaded from the following
link:
\begin{verbatim}
http://spi.cern.ch/Components/UnitTesting/UserDocumentation/Web/Testing_examples.tar.gz
\end{verbatim}

To write your first unit tests with MS Visual C++, you have to do the following steps:
\begin{itemize}
\item RTTI must be enabled (Project : Settings : C++ : C++ Language)
\item link against \textit{lib/cppunitXX.lib} (XX for the chosen library)
\item add the include directory of your cppunit directory to the search path
\end{itemize}

\section{Example}

Here is a little example for a general unit test coded in plain C,
so that you understand the idea how unit tests work.

We are programming a small function which is just adding two
numbers.

\begin{verbatim}
int addition(int a, int b) {
    return (a + b);
}
\end{verbatim}

A second module (also a C-function), is our first test case. It
tests all possible addition cases and compares the result with the
expected value.

\begin{verbatim}
BOOL additionTest() {
    if ( addition(1, 2) != 3 )
        return (FALSE);
    if ( addition(0, 0) != 0 )
        return (FALSE);
    if ( addition(10, 0) != 10 )
        return (FALSE);
    if ( addition(-8, 0) != -8 )
        return (FALSE);
    if ( addition(5, -5) != 0 )
        return (FALSE);
    if ( addition(-5, 2) != -3 )
        return (FALSE);
    if ( addition(-4, -1) != -5 )
        return (FALSE);
    return (TRUE);
}
\end{verbatim}

A second test case (another C-function) tests the behavior of our
module with the typical addition properties.

\begin{verbatim}

BOOL additionPropertiesTest() {
    \textit{// conmutative: a + b = b + a}
    if ( addition(1, 2) != addition(2, 1) )
        return (FALSE);
    \textit{// asociative: a + (b + c) = (a + b) + c}
    if ( addition(1, addition(2, 3)) != addition(addition(1, 2), 3) )
        return (FALSE);
    \textit{// neutral element: a + NEUTRAL = a}
    if ( addition(10, 0) != 10 )
        return (FALSE);
    \textit{// inverse element: a + INVERSE = NEUTRAL}
    if ( addition(10, -10) != 0 )
        return (FALSE);
    return (TRUE);
}
\end{verbatim}

We also can write more test cases for checking other scenarios.

For special examples about unit tests with CPPUnit please check
the link mentioned in the section \textit{Creating Unit Tests}.
