\section{System Overview}

To get started you should be somewhat familiar with the components of
the system. The soccer simulation consists of three important parts:
the server, the monitor and the agents.

\subsection{Server}

In order to work with the server you should be familiar with the
SPADES~\cite{RR03}\cite{SPADES} simulation middleware. Some important
concepts you should know about: The server is responsible to start an
agent process, i.e. it does not wait for an agent to connect as the 2D
simulation does. The SPADES library uses a database that contains
information how to start different agent types. It is called
\texttt{agentdb.xml}, located in the \texttt{./app/simulator/}
directory.

Agents connect via UNIX pipes to a SPADES Commserver. The use a length
prefixed format to exchange messages. The Commserver in turn
communicates with the server. In the default setup of the soccer
server an integrated Commserver is started.

It is to possible start more than one Commserver in order to
distribute agent processes across different systems. Please see the
SPADES manual for further details about how to start and configure a
remote Commserver. In this setup the 3d server has to be configured to
wait until all Commservers are connected before it unpauses the
simulation. The relevant settings are found in the server startup
script \texttt{rcssserver3D.rb}.

These settings are 'Spades.RunIntegratedCommserver' and
'Spades.CommServersWanted'. The first setting configures if the
integrated Commserver is started. Its default value is 'true'. The
second setting gives the number of Commservers the server will wait
for, before the simulation is initially unpaused. The integrated
Commserver counts as one, so the default value here is 1.

\subsection{Monitor}

The default monitor is called \texttt{rcssmonitor3D-lite}, located in
the directory \texttt{./app/rcssmonitor3d/lite}. It is also used to
replay logfiles that the server automatically creates (use the
\texttt{--logfile <filename>} option). The automatically generated
logfile is called 'monitor.log'. You'll find it in the
\texttt{Logfiles/} directory below the directory in which you started
the server. A set of logfiles from 2004 RoboCup can be found
at~\cite{RC04LOGS}.

The implemented monitor protocol supports a command set to implement a
trainer, i.e to automatically recreate test situations on the field
and to evaluate an agents behavior. A 'monitor library' is provided to
help implementing custom monitor and trainer applications, please see
the \texttt{./app/rcssmonitor3d/lib} directory. The protocol between
server and monitor is detailed further down in this text file.

A good starting point for your own agent implementations is the
\texttt{agenttest} program in the \texttt{./app/agenttest/} directory.
This agent implements a simple kick and run behavior.

%%% Local Variables: 
%%% mode: latex
%%% TeX-master: "manual"
%%% TeX-master: t
%%% End: 
