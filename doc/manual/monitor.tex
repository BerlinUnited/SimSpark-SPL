\section{Monitor and Trainer Protocol}

The default monitor port for the soccer simulation is 12001. The
server periodically sends you lines of text that contain
S-Expressions. The monitor log file, that contains the recorded
sequence of all expressions sent to the monitor is further used as the
log file format. It is automatically generated in
\texttt{Logfiles/monitor.log} relative to the server directory.

\subsection{init Expression}

Initially one \texttt{Init} expression is sent. An example
\texttt{init} expression is given below. Note that S-Expressions from
the server are received as a single line. Their are reformatted here
for readability.

\begin{verbatim}
(Init 
      (FieldLength 104)(FieldWidth 68)(FieldHeight 40)
      (GoalWidth 7.32)(GoalDepth 2)(GoalHeight 0.5)(BorderSize 10)
      (FreeKickDistance 9.15)(WaitBeforeKickOff 2)(AgentMass 75)
      (AgentRadius 0.22)(AgentMaxSpeed 10)(BallRadius 0.111)
      (BallMass 0.425878)(RuleGoalPauseTime 3)(RuleKickInPauseTime 1)
      (RuleHalfTime 300)
      (play_modes BeforeKickOff KickOff_Left KickOff_Right PlayOn
      KickIn_Left KickIn_Right corner_kick_left corner_kick_right
      goal_kick_left goal_kick_right offside_left offside_right
      GameOver Goal_Left Goal_Right free_kick_left free_kick_right)
      )
\end{verbatim}

Each subexpression of the init expression is a name value pair that
gives one parameter that the current instance of the simulation uses.
The meaning of the different parameters:

\begin{itemize}
  
\item \texttt{FieldLength},\texttt{FieldWidth},\texttt{FieldHeight}:
  dimensions of the soccer field in meter
  
\item \texttt{GoalWidth}, \texttt{GoalDepth}, \texttt{GoalHeight}:
  dimensions of the goals in meter
  
\item \texttt{BorderSize}: the simulated soccer field is surrounded by
  an off field area. \texttt{BorderSize} gives the extra space in
  meters relative to the regular field dimensions in meters
  
\item \texttt{FreeKickDistance}: gives the distance in meters that
  agents of the opposite have to adhere when a player carries out a
  free kick.
  
\item \texttt{WaitBeforeKickOff}: gives the time in seconds the server
  waits before automatically starting the game

\item \texttt{AgentMass}: the mass of each agent in \texttt{kg}

\item \texttt{AgentRadius}: the radius of each agent in \texttt{m}
  
\item \texttt{AgentMaxSpeed}: the maximum speed of each agent in
  \texttt{m/s}

\item \texttt{BallRadius}: the radius if the ball in \texttt{m}
  
\item \texttt{BallMass}: the mass of the ball in \texttt{kg}
  
\item \texttt{RuleGoalPauseTime}: the time in seconds that the server
  waits after a goal is scored before switching to kick off playmode
  
\item \texttt{RuleKickInPauseTime}: the time in seconds that the
  server waits after the ball left the field before switching to the
  kick in playmode

\item \texttt{RuleHalfTime}: the length of one half time in seconds
  
\item \texttt{play\_modes}: lists the the different
  \texttt{play\_modes} of the soccer simulation. Later on
  \texttt{play\_modes} are referenced by a zero based index into this
  list.

\end{itemize}


\subsection{info Expression}

After the initial init message is sent only Info expressions are sent.
These expressions contain the full state of the current simulation
state. An example \texttt{Info} expression is given below:

\begin{verbatim}
(Info (time 0)(half 1)(score_left 0)(score_right 0)(play_mode 0) (P
(pos 0 0 0))(P (pos 0 0 0))(P (pos 0 0 0))(P (pos 0 0 0)) (P (pos 0 0
0))(P (pos 0 0 0))(P (pos 0 0 0))(P (pos 0 0 0)) (P (pos 0 0 0))(P
(pos 0 0 0))(F (id 1_l)(pos -52 -34 0)) (F (id 2_l)(pos -52 34 0))(F
(id 1_r)(pos 52 -34 0))(F (id 2_r)(pos 52 34 0)) (G (id 1_l)(pos -52
-3.66 0))(G (id 2_l)(pos -52 3.66 0)) (G (id 1_r)(pos 52 -3.66 0))(G
(id 2_r)(pos 52 3.66 0)) (B (pos 0 0 10)) )
\end{verbatim}

Each subexpression of the \texttt{info} expression is a name value
pair that contains information about one aspect of the current
simulation state.  Not all subexpressions are repeated. This concerns
the positions of the field flags and the names of the two teams. This
information is only sent once. Further game state information like the
score count, and the current game state is only sent if it changed.
The meaning of the different expressions:

\begin{itemize}
  
\item \texttt{Die}: notifies the monitor that the soccer simulation is about to
  terminate
  
\item \texttt{time}: the current simulation time in seconds
  
\item \texttt{half}: the current game half, 0 means the first, 1 means the
  second game half
  
\item \texttt{score\_left}, \texttt{score\_right}: the score count of
  the left and right team respectively
  
\item \texttt{team\_left}, \texttt{team\_right}: gives the names of the
  left and right team respectively; the information is only sent once
  as it remains static
  
\item \texttt{play\_mode}: the current play mode as 0 based index into
  the \texttt{play\_modes} list given in the init expression
  
\item \texttt{P}: gives information about a player. This expression
  may contain further subexpressions.  

  \begin{itemize}
  \item \texttt{s}: gives the team the player belongs to; 0 for the
    left, 1 for the right team
    
  \item \texttt{id}: gives the uniform number of the player
    
  \item \texttt{pos}: gives the position of the player as a three
    component vector
    
  \item \texttt{last}: if this subexpression is present, the player
    was the last to touch the ball 
    
  \item \texttt{say}: this expression gives the string the player sent
    using the optional \texttt{SayEffector}

  \end{itemize}
  
\item \texttt{F}: gives information about a flag on the field.
  Information about a flag is only sent once, as it remains static 

  \begin{itemize}
    
  \item \texttt{pos}: gives the position of the flag as a three
    component vector
    
  \item \texttt{id}: gives the name of the flag
    
  \end{itemize}
  
\item \texttt{B}: gives information about the ball 

  \begin{itemize}
    
  \item \texttt{pos}: gives the position of the ball as a three
    component vector
    
  \end{itemize}
  
\item \texttt{ack}: acknowledges a command that is carried out by the
  server; carries a user defined cooky string as parameter; see below
  for further explanation

\end{itemize}

\subsection{Monitor Command Parser}

A connected monitor can further send commands as S-Expressions to the
server using the monitor connection. These commands allow a connected
monitor to set the current playmode and to move players and the ball
to arbitrary positions on the field. This allows for the
implementation of trainer clients.

Supported expressions are:

\begin{itemize}
  
\item \texttt{(kickOff)}: start the soccer game, tossing a coin to
  select the team that kicks off first
  
\item \texttt{(playMode <play\_mode>)}: set the current playmode.
  Possible playmodes are given as strings in the \texttt{play\_modes}
  expression of the init expression the monitor receives when it
  connects.  Example: \texttt{(playMode corner\_kick\_left)}
  
\item \texttt{(agent(team [R,L])(unum <uniform number>(pos
    <x,y,z>)(vel <vx,vy,vz))}. This expression sets the position and
  velocity of the given player on the field. Example: \texttt{(agent
    (team L)(unum 1)(pos -52.0 0.0 0.3)(vel 0.0 0.0 0.0))}
  
\item \texttt{(ball (pos <x,y,z>))}: set the position of the ball on
  the field. Example: \texttt{(ball (pos 10,20,1))}
  
\item \texttt{(dropBall)}: drop ball at its current position and move
  all players away by the free kick radius.
  
\item \texttt{(getAck <cooky string>)}: experimental feature,
  currently disabled. Requests an \texttt{(ack <cooky>)} reply from
  the server.  The server will send the answer as soon as the command
  is carried out. This is used to synchronize a trainer implementation
  wit the server. The getAck expression is appended behind one of the
  above commands. Example: \texttt{((kickOff)(getAck kicked\_off))}

\end{itemize}


%%% Local Variables: 
%%% mode: latex
%%% TeX-master: "manual"
%%% TeX-master: t
%%% End: 




